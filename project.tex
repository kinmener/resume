
\myscitem{Selected Research Projects}

\begin{myexp}
%\item \textbf{Research Assistant to Professor Vijay K. Garg, UT Austin}   \myrightdate{Aug.~2012 -- Present}
\item \textbf{The University of Texas at Austin}   \myrightdate{Sept.~2010 -- Present}
    \begin{myexp}
    \item {\bf Active monitor (Java)}:
            Developed the framework of the active monitor in Java. Monitors are
            the prevalent programming technique for synchronization between
            threads in shared-memory parallel programs. We introduce constructs
            that allow programmers to easily write monitor-based multi-threaded
            programs with performance gains by increasing parallelism in
            monitor objects.
        %\begin{mybullet}
            %\item Designed and implemented the framework of the active 
                %monitor in Java
            %\item Compared performances with explicit-signal monitor
            %\item Accepted by the {\it ACM SIGPLAN conference on Programming Language Design and Implementation (PLDI'13)}
        %\end{mybullet}

    \item {\bf Automatic-signal monitor (Java)}: 
        Developed the framework of the automatic-signal monitor in Java. Most
        programming languages use monitors with explicit signals for
        synchronization in shared-memory programs. Requiring programmers to
        signal threads explicitly results in many concurrency bugs. We
        introduce three novel ideas, {\it closure}, {\it relay invariance}, and
        {\it predicate tagging} to show that the common belief that automatic 
        signaling monitors are prohibitively expensive is wrong. 
            %\item Designed and implemented the framework of the
            %automatic-signal
                %monitor in Java
        %\begin{mybullet}
            %\item Designed and implemented the framework of the automatic-signal
                %monitor in Java
            %\item Compared performances with explicit-signal monitor
            %\item Accepted by the {\it ACM SIGPLAN conference on Programming Language Design and Implementation (PLDI'13)}
        %\end{mybullet}
    %\end{myexp}
%\item {\bf Research Project with Professor Michael Walfish,  UT Austin}  \myrightdate{Dec.~2010 -- May 2011}   
    %\begin{myexp}
    \item {\bf Failure detector in distributed systems (Java)}:
        Implemented the Falcon failure detector in Java.
        A common way for a distributed system to tolerate crashes is to
        explicitly detect them and then recover from them. Current approaches 
        use end-to-end timeout to detect failures in distributed systems. As a
        result, detection can take much longer than recovery. We coordinate a 
        network of spies, each monitoring a layer of the system, to make our
        failure detector efficient and reliable with only slight cost. 
        %\begin{mybullet}
            %\item Implemented the Falcon failure detector in Java
            %\item Evaluated the performances of the Falcon failure detector
            %\item Published in the {\it ACM Symposium on Operating Systems Principles (SOSP'11)}
        %\end{mybullet}
%\item {\bf Research Project with Professor Emmett Witchel ,  The University of Texas at Austin} \\ \myrightdate{Sep.~2010 -- Jan.~2011}   
%\item {\bf Research Project with Professor Emmett Witchel ,  The University of Texas at Austin} \\ \myrightdate{Sep.~2010 -- Jan.~2011}   
    %\item Composing linked list operations without locks  
        %\begin{mybullet}
            %\item Designed and implemented an optimistic lock-free concurrent list in Java
            %\item Compared performances with other lock and lock-free list implementations
        %\end{mybullet}
\end{myexp}
%\item \textbf{Research Assistant to Professor Jie-Hong Jiang, National Taiwan University}   \myrightdate{May 2006 -- July 2010}
%\item \textbf{National Taiwan University, Taiwan}   \myrightdate{May 2006 -- July 2010}
%%\begin{myexp}
%    \begin{myexp}
%    %\item Quantified Boolean formula (QBF) solver 
%        %\begin{mybullet}
%            %\item Designed and implemented the AIG-based QBF solver
%            %\item Compared performances with prior QBF solvers
%            %\item Taught graduate students programming tools
%            %\item Sponsored by the National Science Council, Taiwan
%        %\end{mybullet}
%    \item {\bf Relation determinization (C/C++)}: 
%        Developed the SAT-based algorithm to extract functions
%        from relation. Boolean relations are an important tool in system
%        synthesis and verification. They can characterize solutions to a set of
%        Boolean constraints. However, for hardware implement of a system, a
%        deterministic function often has to be extracted from a relation. Prior
%        methods focused on function optimization but were unlikely to handle
%        large problem instances. We focused on scalability with reasonable
%        optimization quality. The results showed that Boolean relations with
%        thousands of variables can be determinized inexpensively using Craig
%        Interpolation. With such extended capacity, we would anticipate
%        real-world applications. 
%        %\begin{mybullet}
%            %\item Designed and implemented the SAT-based algorithm to extract functions from relation
%            %\item Prepared real state transition relation benchmark circuits
%            %\item Published in the {\it IEEE/ACM International Conference on Computer-Aided Design (ICCAD'09)}
%        %\end{mybullet}
%    \item {\bf Boolean function bi-decomposition (C/C++)}: 
%        Implemented the algorithm of computing Craig Interpolant for SAT-based
%        function bi-decomposition. Boolean function bi-decomposition is a
%        fundamental operation in logic synthesis. Finding such decomposition
%        educes circuit complexity and yields simple physical design solutions.
%        Prior methods may not be able to decompose large functions. We proposed
%        a novel solution to these difficulties using Craig Interpolation and
%        SAT Solving. The experiments showed promising results on the
%        scalability of bi-decomposition. Thereby the capacity of
%        bi-decomposition can be much extended for large functions.
%        %\begin{mybullet}
%            %\item Implemented the algorithm of computing Craig Interpolant
%            %\item Evaluated the scalability of bi-decomposition and the optimality of variable partition
%            %\item Published in the {\it ACM/IEEE Design Automation Conference (DAC'08)}
%        %\end{mybullet}
%    \item {\bf Sequential circuit verification (C/C++)}: 
%        Developed the SAT-based inductive equivalence checking
%        algorithm. Retiming and resynthesis are among the most important 
%        techniques for practical sequential circuit optimization. However, their
%        applicability is much limited due to verification concerns. To overcome
%        this bottleneck, we studied both the theoretical and practical aspects
%        of inductive verification on the equivalence between circuits under 
%        retiming and resynthesis transformation. Through the study of its
%        completeness condition, we extended the verification capability and 
%        capacity. Our results overcome part of the verification bottleneck in
%        logic synthesis.
%\end{myexp}
\end{myexp}
